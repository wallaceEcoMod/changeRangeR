% Options for packages loaded elsewhere
\PassOptionsToPackage{unicode}{hyperref}
\PassOptionsToPackage{hyphens}{url}
%
\documentclass[
]{article}
\usepackage{lmodern}
\usepackage{amssymb,amsmath}
\usepackage{ifxetex,ifluatex}
\ifnum 0\ifxetex 1\fi\ifluatex 1\fi=0 % if pdftex
  \usepackage[T1]{fontenc}
  \usepackage[utf8]{inputenc}
  \usepackage{textcomp} % provide euro and other symbols
\else % if luatex or xetex
  \usepackage{unicode-math}
  \defaultfontfeatures{Scale=MatchLowercase}
  \defaultfontfeatures[\rmfamily]{Ligatures=TeX,Scale=1}
\fi
% Use upquote if available, for straight quotes in verbatim environments
\IfFileExists{upquote.sty}{\usepackage{upquote}}{}
\IfFileExists{microtype.sty}{% use microtype if available
  \usepackage[]{microtype}
  \UseMicrotypeSet[protrusion]{basicmath} % disable protrusion for tt fonts
}{}
\makeatletter
\@ifundefined{KOMAClassName}{% if non-KOMA class
  \IfFileExists{parskip.sty}{%
    \usepackage{parskip}
  }{% else
    \setlength{\parindent}{0pt}
    \setlength{\parskip}{6pt plus 2pt minus 1pt}}
}{% if KOMA class
  \KOMAoptions{parskip=half}}
\makeatother
\usepackage{xcolor}
\IfFileExists{xurl.sty}{\usepackage{xurl}}{} % add URL line breaks if available
\IfFileExists{bookmark.sty}{\usepackage{bookmark}}{\usepackage{hyperref}}
\hypersetup{
  pdftitle={SingleSpeciesVignette},
  pdfauthor={P Galante},
  hidelinks,
  pdfcreator={LaTeX via pandoc}}
\urlstyle{same} % disable monospaced font for URLs
\usepackage[margin=1in]{geometry}
\usepackage{color}
\usepackage{fancyvrb}
\newcommand{\VerbBar}{|}
\newcommand{\VERB}{\Verb[commandchars=\\\{\}]}
\DefineVerbatimEnvironment{Highlighting}{Verbatim}{commandchars=\\\{\}}
% Add ',fontsize=\small' for more characters per line
\usepackage{framed}
\definecolor{shadecolor}{RGB}{248,248,248}
\newenvironment{Shaded}{\begin{snugshade}}{\end{snugshade}}
\newcommand{\AlertTok}[1]{\textcolor[rgb]{0.94,0.16,0.16}{#1}}
\newcommand{\AnnotationTok}[1]{\textcolor[rgb]{0.56,0.35,0.01}{\textbf{\textit{#1}}}}
\newcommand{\AttributeTok}[1]{\textcolor[rgb]{0.77,0.63,0.00}{#1}}
\newcommand{\BaseNTok}[1]{\textcolor[rgb]{0.00,0.00,0.81}{#1}}
\newcommand{\BuiltInTok}[1]{#1}
\newcommand{\CharTok}[1]{\textcolor[rgb]{0.31,0.60,0.02}{#1}}
\newcommand{\CommentTok}[1]{\textcolor[rgb]{0.56,0.35,0.01}{\textit{#1}}}
\newcommand{\CommentVarTok}[1]{\textcolor[rgb]{0.56,0.35,0.01}{\textbf{\textit{#1}}}}
\newcommand{\ConstantTok}[1]{\textcolor[rgb]{0.00,0.00,0.00}{#1}}
\newcommand{\ControlFlowTok}[1]{\textcolor[rgb]{0.13,0.29,0.53}{\textbf{#1}}}
\newcommand{\DataTypeTok}[1]{\textcolor[rgb]{0.13,0.29,0.53}{#1}}
\newcommand{\DecValTok}[1]{\textcolor[rgb]{0.00,0.00,0.81}{#1}}
\newcommand{\DocumentationTok}[1]{\textcolor[rgb]{0.56,0.35,0.01}{\textbf{\textit{#1}}}}
\newcommand{\ErrorTok}[1]{\textcolor[rgb]{0.64,0.00,0.00}{\textbf{#1}}}
\newcommand{\ExtensionTok}[1]{#1}
\newcommand{\FloatTok}[1]{\textcolor[rgb]{0.00,0.00,0.81}{#1}}
\newcommand{\FunctionTok}[1]{\textcolor[rgb]{0.00,0.00,0.00}{#1}}
\newcommand{\ImportTok}[1]{#1}
\newcommand{\InformationTok}[1]{\textcolor[rgb]{0.56,0.35,0.01}{\textbf{\textit{#1}}}}
\newcommand{\KeywordTok}[1]{\textcolor[rgb]{0.13,0.29,0.53}{\textbf{#1}}}
\newcommand{\NormalTok}[1]{#1}
\newcommand{\OperatorTok}[1]{\textcolor[rgb]{0.81,0.36,0.00}{\textbf{#1}}}
\newcommand{\OtherTok}[1]{\textcolor[rgb]{0.56,0.35,0.01}{#1}}
\newcommand{\PreprocessorTok}[1]{\textcolor[rgb]{0.56,0.35,0.01}{\textit{#1}}}
\newcommand{\RegionMarkerTok}[1]{#1}
\newcommand{\SpecialCharTok}[1]{\textcolor[rgb]{0.00,0.00,0.00}{#1}}
\newcommand{\SpecialStringTok}[1]{\textcolor[rgb]{0.31,0.60,0.02}{#1}}
\newcommand{\StringTok}[1]{\textcolor[rgb]{0.31,0.60,0.02}{#1}}
\newcommand{\VariableTok}[1]{\textcolor[rgb]{0.00,0.00,0.00}{#1}}
\newcommand{\VerbatimStringTok}[1]{\textcolor[rgb]{0.31,0.60,0.02}{#1}}
\newcommand{\WarningTok}[1]{\textcolor[rgb]{0.56,0.35,0.01}{\textbf{\textit{#1}}}}
\usepackage{graphicx,grffile}
\makeatletter
\def\maxwidth{\ifdim\Gin@nat@width>\linewidth\linewidth\else\Gin@nat@width\fi}
\def\maxheight{\ifdim\Gin@nat@height>\textheight\textheight\else\Gin@nat@height\fi}
\makeatother
% Scale images if necessary, so that they will not overflow the page
% margins by default, and it is still possible to overwrite the defaults
% using explicit options in \includegraphics[width, height, ...]{}
\setkeys{Gin}{width=\maxwidth,height=\maxheight,keepaspectratio}
% Set default figure placement to htbp
\makeatletter
\def\fps@figure{htbp}
\makeatother
\setlength{\emergencystretch}{3em} % prevent overfull lines
\providecommand{\tightlist}{%
  \setlength{\itemsep}{0pt}\setlength{\parskip}{0pt}}
\setcounter{secnumdepth}{-\maxdimen} % remove section numbering

\title{SingleSpeciesVignette}
\author{P Galante}
\date{2020-09-04}

\begin{document}
\maketitle

\hypertarget{single-species-range-change-metrics}{%
\subsection{Single species range change
metrics}\label{single-species-range-change-metrics}}

Translating a species' current distribution into meaningful conservation
metrics in a repeatable and transparent way to inform conservation
planning and decision-making remains an outstanding issue in
conservation biology. By using a species distribution model (SDM), as
well as landscape requirements (e.g., forest cover), we can mask the
output of an SDM to only those areas likely to be suitable to estimate
the species' current range (e.g., in
\href{maskRangeR}{https://cmerow.github.io/maskRangeR/}. From these
reduced model outputs, upper bounds of IUCN metrics regarding area of
occupancy (AOO) and extent of occurrence (EOO) can be calculated to
inform the assessment of a species' conservation status, in combination
with other information {[}1{]}. In addition, we can calculate the
proportion of a species' range size that is protected, that is
threatened, or that is associated with different land cover types. If
past or future model projections or geospatial data on habitat for
masking are available, we can also calculate and visualize change in
these metrics over time. These change metrics can then inform IUCN
red-listing and forward-thinking conservation planning. We provide an
example below to calculate these metrics for the olinguito {[}2{]} using
the changeRangeR package. Beyond single species, we can combine models
from multiple species to calculate community-level metrics of
conservation interest to learn more about this see our multi-species
vignettes (Here and Here).

{[}1{]} IUCN Standards and Petitions Committee. 2019. Guidelines for
Using the IUCN Red List Categories and Criteria. Version 14. Prepared by
the Standards and Petitions Committee.

{[}2{]} Helgen, K.M., Miguel Pinto, C., Kays, R., Helgen, L.E.,
Tsuchiya, M. T. N., Quinn, A., Wilson, D.E., Maldonado, J.E. (2013)
Taxonomic revision of the olingos (Bassaricyon), with description of a
new species, the Olinguito. Zookeys, 324, 1-83.
\url{https://doi.org/10.3897/zookeys.324.5827}.

Load the packages you'll need

\begin{Shaded}
\begin{Highlighting}[]
\KeywordTok{library}\NormalTok{(changeRangeR)}
\end{Highlighting}
\end{Shaded}

\begin{verbatim}
## Loading required package: Matrix.utils
\end{verbatim}

\begin{verbatim}
## Loading required package: Matrix
\end{verbatim}

\begin{verbatim}
## Loading required package: parallel
\end{verbatim}

\begin{verbatim}
## Loading required package: raster
\end{verbatim}

\begin{verbatim}
## Loading required package: sp
\end{verbatim}

\begin{verbatim}
## Loading required package: R.utils
\end{verbatim}

\begin{verbatim}
## Loading required package: R.oo
\end{verbatim}

\begin{verbatim}
## Loading required package: R.methodsS3
\end{verbatim}

\begin{verbatim}
## R.methodsS3 v1.8.0 (2020-02-14 07:10:20 UTC) successfully loaded. See ?R.methodsS3 for help.
\end{verbatim}

\begin{verbatim}
## R.oo v1.23.0 successfully loaded. See ?R.oo for help.
\end{verbatim}

\begin{verbatim}
## 
## Attaching package: 'R.oo'
\end{verbatim}

\begin{verbatim}
## The following object is masked from 'package:R.methodsS3':
## 
##     throw
\end{verbatim}

\begin{verbatim}
## The following objects are masked from 'package:raster':
## 
##     extend, trim
\end{verbatim}

\begin{verbatim}
## The following objects are masked from 'package:methods':
## 
##     getClasses, getMethods
\end{verbatim}

\begin{verbatim}
## The following objects are masked from 'package:base':
## 
##     attach, detach, load, save
\end{verbatim}

\begin{verbatim}
## R.utils v2.9.2 successfully loaded. See ?R.utils for help.
\end{verbatim}

\begin{verbatim}
## 
## Attaching package: 'R.utils'
\end{verbatim}

\begin{verbatim}
## The following objects are masked from 'package:raster':
## 
##     extract, resample
\end{verbatim}

\begin{verbatim}
## The following object is masked from 'package:utils':
## 
##     timestamp
\end{verbatim}

\begin{verbatim}
## The following objects are masked from 'package:base':
## 
##     cat, commandArgs, getOption, inherits, isOpen, nullfile, parse,
##     warnings
\end{verbatim}

\begin{verbatim}
## Loading required package: textTinyR
\end{verbatim}

\begin{verbatim}
## Loading required package: tools
\end{verbatim}

\begin{verbatim}
## To get started, with multiple species see the demo with 
## vignette(package='changeRangeR')
\end{verbatim}

\begin{Shaded}
\begin{Highlighting}[]
\KeywordTok{library}\NormalTok{(raster)}
\KeywordTok{library}\NormalTok{(rgeos)}
\end{Highlighting}
\end{Shaded}

\begin{verbatim}
## rgeos version: 0.5-3, (SVN revision 634)
##  GEOS runtime version: 3.8.0-CAPI-1.13.1 
##  Linking to sp version: 1.4-2 
##  Polygon checking: TRUE
\end{verbatim}

\begin{Shaded}
\begin{Highlighting}[]
\KeywordTok{library}\NormalTok{(rgdal)}
\end{Highlighting}
\end{Shaded}

\begin{verbatim}
## rgdal: version: 1.5-16, (SVN revision 1050)
## Geospatial Data Abstraction Library extensions to R successfully loaded
## Loaded GDAL runtime: GDAL 3.0.4, released 2020/01/28
## Path to GDAL shared files: C:/Users/pgalante/Documents/R/win-library/4.0/rgdal/gdal
## GDAL binary built with GEOS: TRUE 
## Loaded PROJ runtime: Rel. 6.3.1, February 10th, 2020, [PJ_VERSION: 631]
## Path to PROJ shared files: C:/Users/pgalante/Documents/R/win-library/4.0/rgdal/proj
## Linking to sp version:1.4-2
## To mute warnings of possible GDAL/OSR exportToProj4() degradation,
## use options("rgdal_show_exportToProj4_warnings"="none") before loading rgdal.
\end{verbatim}

\begin{verbatim}
## 
## Attaching package: 'rgdal'
\end{verbatim}

\begin{verbatim}
## The following object is masked from 'package:R.oo':
## 
##     getDescription
\end{verbatim}

\begin{Shaded}
\begin{Highlighting}[]
\KeywordTok{library}\NormalTok{(sf)}
\end{Highlighting}
\end{Shaded}

\begin{verbatim}
## Linking to GEOS 3.8.0, GDAL 3.0.4, PROJ 6.3.1
\end{verbatim}

\begin{Shaded}
\begin{Highlighting}[]
\KeywordTok{library}\NormalTok{(dplyr)}
\end{Highlighting}
\end{Shaded}

\begin{verbatim}
## 
## Attaching package: 'dplyr'
\end{verbatim}

\begin{verbatim}
## The following objects are masked from 'package:rgeos':
## 
##     intersect, setdiff, union
\end{verbatim}

\begin{verbatim}
## The following objects are masked from 'package:raster':
## 
##     intersect, select, union
\end{verbatim}

\begin{verbatim}
## The following objects are masked from 'package:stats':
## 
##     filter, lag
\end{verbatim}

\begin{verbatim}
## The following objects are masked from 'package:base':
## 
##     intersect, setdiff, setequal, union
\end{verbatim}

\hypertarget{range-size}{%
\section{Range size}\label{range-size}}

Calculating range size is as simple as multiplying the number of cells
in a binary raster by the resolution (in km) squared. This method is
useful when your raster is projected. For unprojected rasters, see
?raster::area

\begin{Shaded}
\begin{Highlighting}[]
\NormalTok{p <-}\KeywordTok{raster}\NormalTok{(}\KeywordTok{paste0}\NormalTok{(}\KeywordTok{system.file}\NormalTok{(}\DataTypeTok{package=}\StringTok{"changeRangeR"}\NormalTok{), }\StringTok{"/extdata/DemoData/SDM/olinguito/Forest_suitable_projected1.tif"}\NormalTok{))}
\CommentTok{# Check that your raster is projected in meters}
\KeywordTok{crs}\NormalTok{(p)}
\end{Highlighting}
\end{Shaded}

\begin{verbatim}
## CRS arguments:
##  +proj=utm +zone=18 +south +datum=WGS84 +units=m +no_defs
\end{verbatim}

\begin{Shaded}
\begin{Highlighting}[]
\CommentTok{# find the number of cells that are not NA}
\NormalTok{pCells <-}\StringTok{ }\KeywordTok{ncell}\NormalTok{(p[}\OperatorTok{!}\KeywordTok{is.na}\NormalTok{(p)])}
\CommentTok{# Convert the raster resolution to km^s}
\NormalTok{Resolution <-}\StringTok{ }\NormalTok{(}\KeywordTok{res}\NormalTok{(p)}\OperatorTok{/}\DecValTok{1000}\NormalTok{)}\OperatorTok{^}\DecValTok{2}
\CommentTok{# Multiply the two}
\NormalTok{area <-}\StringTok{ }\NormalTok{pCells }\OperatorTok{*}\StringTok{ }\NormalTok{Resolution}

\KeywordTok{paste0}\NormalTok{(}\StringTok{"area = "}\NormalTok{, area[}\DecValTok{1}\NormalTok{], }\StringTok{" km^2"}\NormalTok{)}
\end{Highlighting}
\end{Shaded}

\begin{verbatim}
## [1] "area = 56965.3594012526 km^2"
\end{verbatim}

\hypertarget{eoo}{%
\section{EOO}\label{eoo}}

\hypertarget{eoo-occurrences}{%
\subsection{EOO Occurrences}\label{eoo-occurrences}}

IUCN's EOO is defined as the area contained within the shortest
imaginary (continuous) boundary drawn to encompass all the known
(current) occurrences of a taxon, excluding vagrant localities. This
measure may exclude discontinuities or disjunctions within the overall
distribution of a taxon (e.g., large areas of unsuitable habitat, but
see AOO below). The EOO is typically measured by drawing a minimum
convex polygon (MCP, also called a convex hull) around occurrence
localities, but this may include many large areas of obviously
unsuitable or unoccupied habitat, making a convex hull around a
thresholded SDM more appropriate (cite Jamie's paper?). It is important
to follow the guidelines of the relevant IUCN SSC SG when contributing
EOO or AOO measurements to enable consistency across assessments. You
can read more about IUCN definitions
\href{https://www.iucnredlist.org/resources/categories-and-criteria}{here}.
In changeRangeR, users can calculate IUCN's EOO via two options 1)
MCP/convex hull around occurrence localities, 2) MCP/convex hull area of
thresholded (MTP) SDM.

Calculate the extent of occupancy around occurrence localities

\begin{Shaded}
\begin{Highlighting}[]
\NormalTok{locs <-}\StringTok{ }\KeywordTok{read.csv}\NormalTok{(}\KeywordTok{paste0}\NormalTok{(}\KeywordTok{system.file}\NormalTok{(}\DataTypeTok{package=}\StringTok{"changeRangeR"}\NormalTok{), }\StringTok{"/extdata/DemoData/locs/10KM_thin_2017.csv"}\NormalTok{))}
\CommentTok{# Look at the first 5 rows. Not that there are three columns: Species, Longitude, Latitude}
\KeywordTok{head}\NormalTok{(locs)}
\end{Highlighting}
\end{Shaded}

\begin{verbatim}
##               species     long     lat
## 1 Bassaricyon neblina -76.5815  3.5009
## 2 Bassaricyon neblina -75.4988  6.2083
## 3 Bassaricyon neblina -78.7491 -0.4550
## 4 Bassaricyon neblina -76.1152  6.3753
## 5 Bassaricyon neblina -76.4453  1.9270
## 6 Bassaricyon neblina -76.8830  2.5330
\end{verbatim}

\begin{Shaded}
\begin{Highlighting}[]
\CommentTok{# Create a minimum convex polygon around the occurrences}
\NormalTok{eoo <-}\StringTok{ }\KeywordTok{mcp}\NormalTok{(locs[,}\DecValTok{2}\OperatorTok{:}\DecValTok{3}\NormalTok{])}
\end{Highlighting}
\end{Shaded}

\begin{verbatim}
## WARNING: this minimum convex polygon has no coordinate reference system.
\end{verbatim}

\begin{Shaded}
\begin{Highlighting}[]
\CommentTok{# Define the coordinate reference system as unprojected}
\KeywordTok{crs}\NormalTok{(eoo) <-}\StringTok{ "+proj=longlat +ellps=WGS84 +datum=WGS84 +no_defs"}
\NormalTok{area <-}\StringTok{ }\KeywordTok{area}\NormalTok{(eoo)}\OperatorTok{/}\DecValTok{1000000}
\CommentTok{## area is measured in meters^2}
\KeywordTok{paste0}\NormalTok{(area, }\StringTok{" km ^2"}\NormalTok{)}
\end{Highlighting}
\end{Shaded}

\begin{verbatim}
## [1] "91721.7270949235 km ^2"
\end{verbatim}

\hypertarget{eoo-sdm}{%
\subsection{EOO SDM}\label{eoo-sdm}}

Calculate the extent of occupancy from a thresholded SDM

\begin{Shaded}
\begin{Highlighting}[]
\NormalTok{p <-}\StringTok{ }\KeywordTok{raster}\NormalTok{(}\KeywordTok{paste0}\NormalTok{(}\KeywordTok{system.file}\NormalTok{(}\DataTypeTok{package=}\StringTok{"changeRangeR"}\NormalTok{), }\StringTok{"/extdata/DemoData/SDM/olinguito/Climatically_suitable_projected1.tif"}\NormalTok{))}
\CommentTok{# Threshold of the minimum training presence}
\NormalTok{thr <-}\StringTok{ }\KeywordTok{min}\NormalTok{(}\KeywordTok{values}\NormalTok{(p), }\DataTypeTok{na.rm=}\NormalTok{T)}
\NormalTok{p[p}\OperatorTok{<}\NormalTok{thr] <-}\StringTok{ }\OtherTok{NA}
\NormalTok{p.pts <-}\StringTok{ }\KeywordTok{rasterToPoints}\NormalTok{(p)}
\NormalTok{eooSDM <-}\StringTok{ }\KeywordTok{mcp}\NormalTok{(p.pts[,}\DecValTok{1}\OperatorTok{:}\DecValTok{2}\NormalTok{])}
\end{Highlighting}
\end{Shaded}

\begin{verbatim}
## WARNING: this minimum convex polygon has no coordinate reference system.
\end{verbatim}

\begin{Shaded}
\begin{Highlighting}[]
\NormalTok{aeoosdm <-}\StringTok{ }\KeywordTok{area}\NormalTok{(eooSDM)}\OperatorTok{/}\DecValTok{1000000}
\KeywordTok{paste0}\NormalTok{(aeoosdm, }\StringTok{" meters ^2"}\NormalTok{)}
\end{Highlighting}
\end{Shaded}

\begin{verbatim}
## [1] "306279.933959894 meters ^2"
\end{verbatim}

\hypertarget{aoo}{%
\section{AOO}\label{aoo}}

Within the calculated EOO area above, users can calculate the sum of 2x2
km grid cells to calculate the upper bounds of IUCN's area of occupancy
or AOO. AOO is intended to account for unsuitable or unoccupied habitats
that may be included in the EOO calculations. AOO should be calculated
with a standard grid cell size of 2 km (a cell area of 4 km2) in order
to ensure consistency and comparability of results in IUCN assessments.
In changeRanger, users can calculate AOO either 1) with occurrence
points, 2) from the pre-masked thresholded SDM, and 3) from the masked
thresholded SDM. It is suggested that users reproject ranges to UTM or
another equal area projection for more accurate area-based calculations.

Calculating the areas of occupancy measured in gridcells where the
resolution is 2km

\hypertarget{aoo-occurrence-points}{%
\subsection{AOO occurrence points}\label{aoo-occurrence-points}}

Calculate the area of occupancy that contains occurrence records

\begin{Shaded}
\begin{Highlighting}[]
\NormalTok{p <-}\KeywordTok{raster}\NormalTok{(}\KeywordTok{paste0}\NormalTok{(}\KeywordTok{system.file}\NormalTok{(}\DataTypeTok{package=}\StringTok{"changeRangeR"}\NormalTok{), }\StringTok{"/extdata/DemoData/SDM/olinguito/Climatically_suitable_projected1.tif"}\NormalTok{))}
\CommentTok{# Using unfiltered records}
\NormalTok{locs <-}\StringTok{ }\KeywordTok{read.csv}\NormalTok{(}\KeywordTok{paste0}\NormalTok{(}\KeywordTok{system.file}\NormalTok{(}\DataTypeTok{package=}\StringTok{"changeRangeR"}\NormalTok{), }\StringTok{"/extdata/DemoData/locs/All_localities_30n.csv"}\NormalTok{))}
\NormalTok{locs <-}\StringTok{ }\NormalTok{locs[,}\DecValTok{1}\OperatorTok{:}\DecValTok{2}\NormalTok{]}
\NormalTok{p[}\OperatorTok{!}\KeywordTok{is.na}\NormalTok{(p)] <-}\StringTok{ }\DecValTok{1}
\NormalTok{AOOlocs<-}\KeywordTok{aooArea}\NormalTok{(}\DataTypeTok{r =}\NormalTok{ p,}\DataTypeTok{proj =} \KeywordTok{crs}\NormalTok{(p), }\DataTypeTok{locs =}\NormalTok{ locs)}

\KeywordTok{print}\NormalTok{(AOOlocs)}
\end{Highlighting}
\end{Shaded}

\begin{verbatim}
## [1] "AOO of cells with occurrence records:120km^2"
\end{verbatim}

\hypertarget{aoo-pre-masked-sdm}{%
\subsection{AOO pre-masked SDM}\label{aoo-pre-masked-sdm}}

\begin{Shaded}
\begin{Highlighting}[]
\NormalTok{p <-}\StringTok{ }\KeywordTok{raster}\NormalTok{(}\KeywordTok{paste0}\NormalTok{(}\KeywordTok{system.file}\NormalTok{(}\DataTypeTok{package=}\StringTok{"changeRangeR"}\NormalTok{), }\StringTok{"/extdata/DemoData/SDM/olinguito/Climatically_suitable_projected1.tif"}\NormalTok{))}
\CommentTok{# Convert to binary}
\NormalTok{p[}\OperatorTok{!}\KeywordTok{is.na}\NormalTok{(p)] <-}\StringTok{ }\DecValTok{1}
\NormalTok{AOO<-}\KeywordTok{aooArea}\NormalTok{(}\DataTypeTok{r =}\NormalTok{ p,}\DataTypeTok{proj =} \KeywordTok{crs}\NormalTok{(p))}
\KeywordTok{print}\NormalTok{(AOO)}
\end{Highlighting}
\end{Shaded}

\begin{verbatim}
## [1] "AOO: 130052 km^2"
\end{verbatim}

\hypertarget{aoo-masked-sdm}{%
\subsection{AOO masked SDM}\label{aoo-masked-sdm}}

\begin{Shaded}
\begin{Highlighting}[]
\NormalTok{p <-}\StringTok{ }\KeywordTok{raster}\NormalTok{(}\KeywordTok{paste0}\NormalTok{(}\KeywordTok{system.file}\NormalTok{(}\DataTypeTok{package=}\StringTok{"changeRangeR"}\NormalTok{), }\StringTok{"/extdata/DemoData/SDM/olinguito/Forest_suitable_projected1.tif"}\NormalTok{))}
\CommentTok{# Convert to binary}
\NormalTok{p[}\OperatorTok{!}\KeywordTok{is.na}\NormalTok{(p)] <-}\StringTok{ }\DecValTok{1}
\NormalTok{AOO<-}\KeywordTok{aooArea}\NormalTok{(}\DataTypeTok{r =}\NormalTok{ p,}\DataTypeTok{proj =} \KeywordTok{crs}\NormalTok{(p))}
\KeywordTok{print}\NormalTok{(AOO)}
\end{Highlighting}
\end{Shaded}

\begin{verbatim}
## [1] "AOO: 103576 km^2"
\end{verbatim}

\hypertarget{optimized-model-threshold}{%
\section{Optimized Model Threshold}\label{optimized-model-threshold}}

Choice of model threshold can have downstream implications for
calculations of metrics such as IUCN's EOO and AOO, when calculated
using SDM inputs. changeRanger includes a function for users to choose
model threshold

Determining the best threshold and area for the SDM. For each increment
of 0.01 between a user-specified threshold and the maximum SDM
prediction value, the prediction is thresholded to this value to make a
binary raster. This raster is then converted to points, which are used
to delineate a trial MCP. Each trial MCP is spatially intersected with
the original MCP (based on the occurrence coordinates) and the original
occurrence points. The Jaccard similarity index is calculated to
determine geographic similarity between the trial and observed MCP. The
trial MCP is also spatially intersected with the original occurrence
points to determine how many were omitted. The ``best'' MCP is the one
that has the highest JSI and also omits the least original occurrence
points.

\begin{Shaded}
\begin{Highlighting}[]
\NormalTok{p <-}\StringTok{ }\KeywordTok{raster}\NormalTok{(}\KeywordTok{paste0}\NormalTok{(}\KeywordTok{system.file}\NormalTok{(}\DataTypeTok{package=}\StringTok{"changeRangeR"}\NormalTok{), }\StringTok{"/extdata/DemoData/SDM/olinguito/olinguitoSDM.tif"}\NormalTok{))}
\NormalTok{xy <-}\StringTok{ }\KeywordTok{read.csv}\NormalTok{(}\KeywordTok{paste0}\NormalTok{(}\KeywordTok{system.file}\NormalTok{(}\DataTypeTok{package=}\StringTok{"changeRangeR"}\NormalTok{), }\StringTok{"/extdata/DemoData/locs/10KM_thin_2017.csv"}\NormalTok{))}
\NormalTok{ch.orig <-}\StringTok{ }\KeywordTok{mcp}\NormalTok{(xy[,}\DecValTok{2}\OperatorTok{:}\DecValTok{3}\NormalTok{])}
\end{Highlighting}
\end{Shaded}

\begin{verbatim}
## WARNING: this minimum convex polygon has no coordinate reference system.
\end{verbatim}

\begin{Shaded}
\begin{Highlighting}[]
\NormalTok{thr <-}\StringTok{ }\FloatTok{0.3380209}
\NormalTok{SDMeoo <-}\StringTok{ }\KeywordTok{mcpSDM}\NormalTok{(}\DataTypeTok{p =}\NormalTok{ p, }\DataTypeTok{xy =}\NormalTok{ xy[,}\DecValTok{2}\OperatorTok{:}\DecValTok{3}\NormalTok{], }\DataTypeTok{ch.orig =}\NormalTok{ ch.orig, }\DataTypeTok{thr =}\NormalTok{ thr)}
\CommentTok{# Check the output}
\NormalTok{SDMeoo}
\end{Highlighting}
\end{Shaded}

\begin{verbatim}
## $jsi
##  [1] 0.05001384 0.05001759 0.05004639 0.05011180 0.05068842 0.05070592
##  [7] 0.05108942 0.05114548 0.05408432 0.05646642 0.05692540 0.05699842
## [13] 0.05703103 0.05706109 0.05712634 0.05822613 0.05859980 0.05861419
## [19] 0.05908503 0.05938596 0.05949351 0.05964649 0.05976235 0.05981566
## [25] 0.06029873 0.06067109 0.06077718 0.06097874 0.06106401 0.06126506
## [31] 0.06141915 0.06155827 0.06164195 0.06176097 0.06188650 0.06209165
## [37] 0.06280664 0.06288058 0.06295917 0.06319632 0.06368854 0.06761048
## [43] 0.06783378 0.06806077 0.06815294 0.06852636 0.06866409 0.06894252
## [49] 0.06951222 0.07004717 0.07033226 0.07079381 0.07190832 0.09096866
## [55] 0.09127588 0.09244362 0.03447275 0.03468914 0.03626676 0.03667725
## [61] 0.03790107 0.00000000 0.00000000 0.00000000
## 
## $thr
##  [1] 0.3380209 0.3480209 0.3580209 0.3680209 0.3780209 0.3880209 0.3980209
##  [8] 0.4080209 0.4180209 0.4280209 0.4380209 0.4480209 0.4580209 0.4680209
## [15] 0.4780209 0.4880209 0.4980209 0.5080209 0.5180209 0.5280209 0.5380209
## [22] 0.5480209 0.5580209 0.5680209 0.5780209 0.5880209 0.5980209 0.6080209
## [29] 0.6180209 0.6280209 0.6380209 0.6480209 0.6580209 0.6680209 0.6780209
## [36] 0.6880209 0.6980209 0.7080209 0.7180209 0.7280209 0.7380209 0.7480209
## [43] 0.7580209 0.7680209 0.7780209 0.7880209 0.7980209 0.8080209 0.8180209
## [50] 0.8280209 0.8380209 0.8480209 0.8580209 0.8680209 0.8780209 0.8880209
## [57] 0.8980209 0.9080209 0.9180209 0.9280209 0.9380209 0.9480209 0.9580209
## [64] 0.9680209
## 
## $ov.pts
##  [1] 18 18 18 18 18 18 18 18 18 18 18 18 18 18 18 18 18 18 18 18 18 18 18 18 18
## [26] 18 18 18 18 18 18 18 18 18 18 18 18 18 18 18 18 18 18 18 18 18 18 18 18 18
## [51] 18 18 18 14 14 14  7  7  7  7  7  0  0  0
## 
## $best.fit
## class       : SpatialPolygons 
## features    : 1 
## extent      : -83.89583, -72.0375, -4.6375, 11.17083  (xmin, xmax, ymin, ymax)
## crs         : +proj=longlat +datum=WGS84 +no_defs 
## 
## $best.fit.ind
## [1] 53
\end{verbatim}

\hypertarget{ratio-overlap}{%
\section{Ratio overlap}\label{ratio-overlap}}

The function ratioOverlap allows changeRangeR users to calculate the
proportion overlap of a species' range with other features, for example
different land cover classes, habitat types, or ecoregions, different
types of threats (any user-defined georeferenced polygon). In this
example, we calculate the proportion of the Olinguito distribution that
overlaps with protected areas in Colombia. NOTE: the protected areas can
be separated by any fields' categories in a shapefile's attribute table.

\hypertarget{current}{%
\subsection{Current}\label{current}}

\begin{Shaded}
\begin{Highlighting}[]
\NormalTok{r <-}\StringTok{ }\KeywordTok{raster}\NormalTok{(}\KeywordTok{paste0}\NormalTok{(}\KeywordTok{system.file}\NormalTok{(}\DataTypeTok{package=}\StringTok{"changeRangeR"}\NormalTok{), }\StringTok{"/extdata/DemoData/SDM/olinguito/Forest_suitable_projected1.tif"}\NormalTok{))}
\NormalTok{shp <-}\StringTok{ }\KeywordTok{readOGR}\NormalTok{(}\DataTypeTok{dsn =} \KeywordTok{paste0}\NormalTok{(}\KeywordTok{system.file}\NormalTok{(}\DataTypeTok{package=}\StringTok{"changeRangeR"}\NormalTok{), }\StringTok{"/extdata/DemoData/shapefiles"}\NormalTok{), }\StringTok{"WDPA_COL_olinguito"}\NormalTok{)}
\end{Highlighting}
\end{Shaded}

\begin{verbatim}
## OGR data source with driver: ESRI Shapefile 
## Source: "C:\Users\pgalante\Documents\R\win-library\4.0\changeRangeR\extdata\DemoData\shapefiles", layer: "WDPA_COL_olinguito"
## with 659 features
## It has 5 fields
\end{verbatim}

\begin{Shaded}
\begin{Highlighting}[]
\CommentTok{# View the fields}
\KeywordTok{colnames}\NormalTok{(shp}\OperatorTok{@}\NormalTok{data)}
\end{Highlighting}
\end{Shaded}

\begin{verbatim}
## [1] "NAME"       "ORIG_NAME"  "DESIG"      "DESIG_ENG"  "DESIG_TYPE"
\end{verbatim}

\begin{Shaded}
\begin{Highlighting}[]
\CommentTok{# Pick the field you are interested in}
\NormalTok{field <-}\StringTok{ "DESIG_ENG"}
\NormalTok{category <-}\StringTok{ "All"}
\NormalTok{ratio.Overlap <-}\StringTok{ }\KeywordTok{ratioOverlap}\NormalTok{(}\DataTypeTok{r =}\NormalTok{ r, }\DataTypeTok{shp =}\NormalTok{ shp, }\DataTypeTok{field =}\NormalTok{ field, }\DataTypeTok{category =}\NormalTok{ category)}
\CommentTok{# Look at the range that is protected}
\KeywordTok{plot}\NormalTok{(ratio.Overlap}\OperatorTok{$}\NormalTok{maskedRange)}
\end{Highlighting}
\end{Shaded}

\includegraphics{singleSpeciesMetrics_files/figure-latex/unnamed-chunk-8-1.pdf}

\begin{Shaded}
\begin{Highlighting}[]
\CommentTok{# The proportion of the range that is protected}
\NormalTok{ratio.Overlap}\OperatorTok{$}\NormalTok{ratio}
\end{Highlighting}
\end{Shaded}

\begin{verbatim}
## [1] "Percentage of range within shape is 11.463793337993%"
\end{verbatim}

\hypertarget{future}{%
\subsection{Future}\label{future}}

For users that have information on past environmental conditions or
future scenarios, they can calculate changes in metrics over time and
view a line graph of those changes. For example, the change in
percentage of forest within species' range over time.

\begin{Shaded}
\begin{Highlighting}[]
\CommentTok{# Load shapefile}
\NormalTok{PA <-}\StringTok{ }\KeywordTok{st_read}\NormalTok{(}\DataTypeTok{dsn =} \KeywordTok{paste0}\NormalTok{(}\KeywordTok{system.file}\NormalTok{(}\DataTypeTok{package=}\StringTok{"changeRangeR"}\NormalTok{), }\StringTok{"/extdata/DemoData/shapefiles/vn"}\NormalTok{), }\StringTok{"VN_NRs"}\NormalTok{)}
\end{Highlighting}
\end{Shaded}

\begin{verbatim}
## Reading layer `VN_NRs' from data source `C:\Users\pgalante\Documents\R\win-library\4.0\changeRangeR\extdata\DemoData\shapefiles\vn' using driver `ESRI Shapefile'
## Simple feature collection with 187 features and 28 fields
## geometry type:  MULTIPOLYGON
## dimension:      XY
## bbox:           xmin: 102.1522 ymin: 8.429265 xmax: 109.4629 ymax: 23.17159
## geographic CRS: WGS 84
\end{verbatim}

\begin{Shaded}
\begin{Highlighting}[]
\CommentTok{# load raster}
\NormalTok{r <-}\StringTok{ }\KeywordTok{stack}\NormalTok{(}\KeywordTok{list.files}\NormalTok{(}\DataTypeTok{path =} \KeywordTok{paste0}\NormalTok{(}\KeywordTok{system.file}\NormalTok{(}\DataTypeTok{package=}\StringTok{"changeRangeR"}\NormalTok{), }\StringTok{"/extdata/DemoData/SDM/franLang"}\NormalTok{), }\DataTypeTok{pattern =} \StringTok{"}\CharTok{\textbackslash{}\textbackslash{}}\StringTok{.tif$"}\NormalTok{, }\DataTypeTok{full.names =}\NormalTok{ T))}
\CommentTok{# Assume PA's will not change, so make list of current protectes areas}
\NormalTok{futures <-}\StringTok{ }\KeywordTok{list}\NormalTok{(PA, PA)}
\CommentTok{# create list of rasters for example}
\NormalTok{r <-}\StringTok{ }\NormalTok{raster}\OperatorTok{::}\KeywordTok{unstack}\NormalTok{(r)}
\CommentTok{# supply names for r and futures}
\NormalTok{r.names <-}\StringTok{ }\KeywordTok{c}\NormalTok{(}\StringTok{"BCC.2040.ssp2"}\NormalTok{, }\StringTok{"BCC.2060.ssp2"}\NormalTok{)}
\NormalTok{futures.names <-}\StringTok{ }\KeywordTok{c}\NormalTok{(}\StringTok{"PA1"}\NormalTok{, }\StringTok{"PA2"}\NormalTok{)}
\CommentTok{# Define shapefile field and category}
\NormalTok{field <-}\StringTok{ "DESIG_ENG"}
\NormalTok{category <-}\StringTok{ "All"}
\CommentTok{# Calculate the overlap for each time period}
\NormalTok{future.ratios <-}\StringTok{ }\KeywordTok{futureOverlap}\NormalTok{(}\DataTypeTok{r =}\NormalTok{ r, }\DataTypeTok{futures =}\NormalTok{ futures, }\DataTypeTok{field =}\NormalTok{ field, }\DataTypeTok{category =}\NormalTok{ category, }\DataTypeTok{futures.names =}\NormalTok{ futures.names, }\DataTypeTok{r.names =}\NormalTok{ r.names)}
\CommentTok{## Plot}
\CommentTok{# Create list of years from which landcover comes}
\NormalTok{years <-}\StringTok{ }\KeywordTok{c}\NormalTok{(}\DecValTok{2040}\NormalTok{, }\DecValTok{2060}\NormalTok{)}
\CommentTok{# Plot}
\KeywordTok{plot}\NormalTok{(}\DataTypeTok{x =}\NormalTok{ years, }\DataTypeTok{y =}\NormalTok{ future.ratios[,}\DecValTok{2}\NormalTok{], }\DataTypeTok{type =} \StringTok{"b"}\NormalTok{, }\DataTypeTok{main =} \StringTok{"Percent of SDM predicted to be protected"}\NormalTok{)}
\end{Highlighting}
\end{Shaded}

\includegraphics{singleSpeciesMetrics_files/figure-latex/unnamed-chunk-9-1.pdf}

\hypertarget{environmental-change-through-time}{%
\subsection{Environmental Change Through
Time}\label{environmental-change-through-time}}

To see how SDM range size can change with suitable forest cover through
time, supply environmental rasters and a suitability threshold as well
as a binary SDM. The environmental rasters must be in the same
coordinate reference system at the SDM.

\begin{Shaded}
\begin{Highlighting}[]
\NormalTok{SDM <-}\StringTok{ }\NormalTok{raster}\OperatorTok{::}\KeywordTok{raster}\NormalTok{(}\KeywordTok{paste0}\NormalTok{(}\KeywordTok{system.file}\NormalTok{(}\DataTypeTok{package=}\StringTok{"changeRangeR"}\NormalTok{), }\StringTok{"/extdata/DemoData/SDM/olinguito/Climatically_suitable_projected1.tif"}\NormalTok{))}
\NormalTok{rStack <-}\StringTok{ }\NormalTok{raster}\OperatorTok{::}\KeywordTok{stack}\NormalTok{(}\KeywordTok{list.files}\NormalTok{(}\DataTypeTok{path =} \KeywordTok{paste0}\NormalTok{(}\KeywordTok{system.file}\NormalTok{(}\DataTypeTok{package=}\StringTok{"changeRangeR"}\NormalTok{), }\StringTok{"/extdata/DemoData/MODIS"}\NormalTok{), }\DataTypeTok{pattern =} \StringTok{"}\CharTok{\textbackslash{}\textbackslash{}}\StringTok{.tif$"}\NormalTok{, }\DataTypeTok{full.names =}\NormalTok{ T))}
\NormalTok{rStack <-}\StringTok{ }\NormalTok{raster}\OperatorTok{::}\KeywordTok{projectRaster}\NormalTok{(rStack, SDM, }\DataTypeTok{method =} \StringTok{'bilinear'}\NormalTok{)}
\NormalTok{threshold <-}\StringTok{ }\FloatTok{50.086735}

\NormalTok{SDM.time <-}\StringTok{ }\KeywordTok{envChange}\NormalTok{(}\DataTypeTok{rStack =}\NormalTok{ rStack, }\DataTypeTok{SDM =}\NormalTok{ SDM, }\DataTypeTok{threshold =}\NormalTok{ threshold)}

\NormalTok{years <-}\StringTok{ }\KeywordTok{c}\NormalTok{(}\StringTok{"2005"}\NormalTok{, }\StringTok{"2006"}\NormalTok{, }\StringTok{"2008"}\NormalTok{, }\StringTok{"2009"}\NormalTok{, }\StringTok{"2010"}\NormalTok{)}

\KeywordTok{plot}\NormalTok{(}\DataTypeTok{y =}\NormalTok{ SDM.time}\OperatorTok{$}\NormalTok{Area, }\DataTypeTok{x =}\NormalTok{ years, }\DataTypeTok{main =} \StringTok{"SDM area change"}\NormalTok{, }\DataTypeTok{ylab =} \StringTok{"area (square m)"}\NormalTok{)}
\KeywordTok{lines}\NormalTok{(}\DataTypeTok{y =}\NormalTok{ SDM.time}\OperatorTok{$}\NormalTok{Area, }\DataTypeTok{x =}\NormalTok{ years)}
\end{Highlighting}
\end{Shaded}

\includegraphics{singleSpeciesMetrics_files/figure-latex/unnamed-chunk-10-1.pdf}

\end{document}
